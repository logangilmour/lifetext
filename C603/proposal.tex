\documentclass{article}
\usepackage{graphicx}
\usepackage{enumerate}

\begin{document}

\title{CMPUT 660 Assignment 1}
\author{Logan Gilmour}

\maketitle

%\begin{abstract}
%The abstract text goes here.
%\end{abstract}

\section{Research Area}
I would like to do my research in Software Engineering / Human-Computer Interaction.

\section{Contacted Researchers}
\begin{enumerate}

  \item Eleni Stroulia, September 23rd, 2013
  \item Victor Guana, September 24th, 2013 
  \item Blerina Bazelli, September 19th, 2013

\end{enumerate}


\section{Research Topic}

I am interested in Human Computation. This is is the problem of designing systems that make use of human effort in order to solve problems that are currently not practical to solve using exclusively computers. Typically, these systems motivate users to contribute their time by presenting the problem to be solved in the form of a game.

This is an interesting area because of the untapped potential that exists. Wikipedia illustrates that, given the proper tools, people will contribute their personal time toward monumental tasks. However, problems with more specific goals seem to require tools that both guide users into producing needed results, and somehow motivate users to do these tasks.

This problem is quite recent, so there are many directions to go. It would be possible to develop and refine methodologies and frameworks for harnessing human computation. It would also be interesting to investigate ways of intermingling AI techniques with human computation. It would also be possible to examine human computation in terms of video-games research. There is also potential for combining this research with pervasive computing research, as there are many tasks involving the real world that typically require human involvement for meaningful results (such as making maps or making aesthetic judgments). 

\section{Relevant Papers}
\subsection{Labeling Images with a Computer Game}

This paper details a computer game that uses human effort to give images labels\cite{vonAhn:2004:LIC:985692.985733}. It is interesting in that it details examples of a number of novel constraints that encourage the desired computation to occur in a form that is fun and rewarding enough for players that they will label the images without compensation.

\subsection{Algorithm discovery by protein folding game players}
Foldit is a game in which players work with proteins\cite{Khatib22112011}. In this paper, players are shown to have collaboratively discovered a protein folding algorithm better than the current published approach, and similar to a new, unpublished approach. Not only are these users not compensated; they are producing work similar to researchers. This result suggests that it is possible to at least partially separate technical expertise from creativity by modeling your problem in such a way that ordinary people can tackle it. 

\subsection{Exploring iterative and parallel human computation processes}

In this paper, researchers explore the differences between human decision and creation tasks accomplished in series versus in parallel\cite{Little:2010:EIP:1837885.1837907}. They use Amazon's Mechanical Turk service to mechanize task delegation. This paper is interesting in that it begins to analyze the ways that human computation can be broken down into different categories, evaluated, and reasoned about. My hope is to contribute to this part of the problem of Human Computation.


\bibliographystyle{plain}
\bibliography{human-computation}
\end{document}
